\documentclass[a4paper, 10pt]{article}
\usepackage[utf8]{inputenc}

\title{Mobile Apps as a Learning Space for Teaching Math: A project on Mexico's public high school system}
\author{Marco Negrete, Javier Alatorre, Daniela Madrigal and Melissa Santoyo}
\date{}

\begin{document}
\maketitle
\begin{abstract}
In Mexico there is a lag in math learning from the basic education to high school and even in higher education. This situation is reflected in the poor performance that Mexican students have in standardized tests, both national and international, where results situate them under the OECD average, showing problems in basic abilities such as expressing problems in mathematical language and relating variables in a given situation. In this work we present a set of learning units focused on the curricula of math courses of a public high school system. This set of learning units is composed of several macro projects where students must apply mathematical high school-level concepts to solve a problem using the same tools that a professional should use. Our main goal is to form a complex learning environment where students assimilate concepts not just by solving exercises or isolated problems, but by getting involved in a human activity of professional level. Our proposal is designed taking as foundation the Leontiev's Activity Theory. We present the results obtained from the first year of application and discuss the future work and expected results. 
\end{abstract}

\section{Introduction}
Justificación (social): bajo rendimiento en los alumnos de bachillerato. Resultados de PISA de 2015 y PLANEA. 

\section{Background}
\cite{freudenthal1973math}: The best way to learn an activity is to perform it. The work of Piaget and his School does not represent an analysis of mathematics as an activity. The re-invention method: interpreting and analyzing mathematics as an activity.  
\cite{freudenthal2002math}: Mathematics is a mental activity. 



Idea sobre razonamiento matemático. Ligar la definición de razonamiento matemático y analizar cómo se reinterpretan los objetivos de aprendizaje de la ENP en el marco del razonamiento matemático. 

Idea de entornos de aprendizaje. Se conjungan la idea del razonamiento matemático con entornos de aprendizaje

\section{Objectives}

\section{Method}

\section{Results and Analysis}

\section{Discussion}

Ideas sobre la matemáticos:
Platón (ideas)
Bertrand Russell
Fregue: Las matemáticas desde el número
Gödel.

%%Cambiar learning unit por learning activities.
%%Cambiar professional por real activities. 

%Las matemáticas son una herramienta de representación simbólica de la realidad. (semiótica matemática) (Paul Ernst)
%Luis Radford. Objetivaciones matemáticas. Las objetivaciones matemáticas permiten pensar colectivamente. 
%Las objetivaciones son el punto de encuentro del sujeto con las matemáticas. 
%Las matemáticas son parte de las actividades reales. 
%Todas las actividades reales sustentadas en la relación con el espacion promueven el pensamiento geométrico. 
%

%objetivo: 

\section{PISA}
\subsection{Definition of mathematical literacy}
Mathematical  literacy  is  an  individual’s  capacity  to  formulate,  employ  and  interpret  mathematics  in  a  variety of  contexts.  It  includes  reasoning  mathematically  and  using  mathematical  concepts,  procedures,  facts  and  tools
to  describe, explain and predict phenomena. It assists individuals to recognise the role that mathematics plays in the world  and  to  make  the  well-founded  judgements  and  decisions  needed  by  constructive,  engaged  and  reflective  citizens \cite{oecd2015framework}.

\subsection{Mathematical processes}
\begin{itemize}
    \item Formulating situations mathematically: being able to recognize and identify opportunities to use mathematics. 
    \item Employing mathematical concepts, facts, procedures and reasoning: able to apply mathematical concepts, facts, procedures and reasoning to solve mathematically formulated problems to obtain mathematical conclusions. 
    \item Interpreting, applying and evaluating mathematical outcomes: being able to reflect upon mathematical solutions, results, or conclusions and interpret them in the context of real-life problems.
\end{itemize}

\subsection{Mathematical capabilities}
\begin{itemize}
    \item Communication
    \item Mathematising
    \item Representation
    \item Reasoning and argument
    \item Devising strategies for solving problems
    \item Using symbolic, formal and technical language and operations
    \item Using mathematical tools
\end{itemize} 

\subsection{Contexts}
\begin{itemize}
    \item Personal: activities of one's self, family or peer group.
    \item Occupational: activities of the world of work.
    \item Societal: activities of one's community whether local, national or global. 
    \item Scientific: application of math to the natural world and issues and topics related to science and technology.
\end{itemize}

\section{Mathematical Modeling}
According to \cite{blum2009mathematical}, mathematical modeling is performed in seven steps:
\begin{enumerate}
    \item Constructing. The problem situation has to be understood and a situation model has to be constructed. The result of this step is a situation model. 
    \item Simplifying/structuring: Once the situation is understood, it has to be simplified and structured to be more precise. The result of this step is a real model. 
    \item Mathematising: The problem solver transforms the real model of the previous step into a mathematical model. 
    \item Working mathematically: Solve equations, make calculations, etc. The result of these step is a set of mathematical results. 
    \item Interpreting: The problem solver translates the mathematical results into real results.
    \item Validating: Checking the results. The result of this step is a decision of going around the loop a second time. 
    \item Exposing: Communicating the results. 
\end{enumerate}

\section{Related Work}
Authors of \cite{bua2016competencia} introduce the concept of praxeology. They state that PISA mathematical capabilities are focused on the \textit{know-how}, but mathematical competences should also include a \textit{know} part, forming so a \textit{praxeology} (praxis + logos). Teaching mathematics should include a practical part (know-how) and a theoretical part (know). 
\bibliographystyle{ieeetr}
\bibliography{References}
\end{document}
